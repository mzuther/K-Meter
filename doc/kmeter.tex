\documentclass[paper=a5,pagesize=auto,twoside=false,fontsize=12pt,DIV=classic,BCOR=0mm,headinclude=true,footinclude=false]{scrartcl} % final

%% font selection
\usepackage{utopia} % final
\usepackage{avant} % final

%% redefine headers
\usepackage[headsepline]{scrpage2}
\pagestyle{scrheadings}

\setkomafont{pageheadfoot}{%
  \normalfont\itshape
}

\automark[section]{chapter}
\renewcommand{\chaptermark}[1]{\markboth{#1}{#1}}
\renewcommand{\sectionmark}[1]{\markright{#1}}

%% additional KOMA script options
\KOMAoptions{parskip=half,headsepline=true,numbers=noendperiod,captions=tableheading}
\recalctypearea{} % recalculate type area after selecting fonts

\usepackage[british]{babel}
\usepackage{graphicx}
\usepackage{ifpdf}
\usepackage[utf8x]{inputenc}
\usepackage{paralist}
\usepackage{ucs}
\usepackage[hyphens]{url}

% Make sure "hyperref" comes last of your loaded packages, to give it
% a fighting chance of not being over-written, since its job is to
% redefine many LATEX commands.
\ifpdf
	\usepackage[unicode,pdftex]{hyperref}
\else
	\usepackage[unicode,hypertex]{hyperref}
\fi

\hypersetup{ % final
  unicode=true,                % non-Latin characters in Acrobat’s bookmarks
  pdftitle=K-Meter,            % title
  pdfauthor=Martin Zuther,     % author
  pdfsubject=K-Meter,          % subject of the document
  pdfcreator=Martin Zuther,    % creator of the document
  colorlinks=true,             % false: boxed links; true: colored links
  linkcolor=blue,              % color of internal links
  citecolor=blue,              % color of links to bibliography
  filecolor=blue,              % color of file links
  urlcolor=blue                % color of external links
}

%% KOMA script captions
\addtokomafont{caption}{\small}
\addtokomafont{captionlabel}{\bfseries}
\setcapindent{0em}

%% taken from package "l2tabu"
\tolerance 1414
\hbadness 1414
\emergencystretch 1.5em
\hfuzz 0.3pt
\widowpenalty=10000
\vfuzz
\hfuzz
\raggedbottom

%% package "paralist"
\setdefaultitem{\textbullet}{$\circ$}{}{}

%% package "url"
\urlstyle{rm}
\input{include/hyphenation.sty}

\title{K-Meter}
\author{Martin Zuther}

\begin{document}

\title{K-Meter}

\subtitle{
  \normalsize{\textrm{\textmd{
        \vfill
        Free implementation of a K-System meter \\
        according to Bob Katz' specifications
        \vfill
        \vspace{1.5em}
        \begin{figure}[H]
          \centering{}
          \includegraphics[scale=0.275,clip]{include/images/kmeter.png}
        \end{figure}
        \vfill
      }}}
}

\author{\normalsize\copyright{} 2010-2012
  \href{http://www.mzuther.de/}{Martin Zuther}}

\date{\normalsize \emph{Last edited on \today}}

\maketitle

\tableofcontents

\clearpage  % layout

\chapter{The loudness race}
\label{chap:loudness_race}

When comparing two similar pieces of music, the louder one is
perceived as sounding better (although this is only true for very
short periods of time).  Accordingly, the loudness of music
productions has continuously grown during the last decades.

As maximum levels of records, tapes and digital media have a natural
limit, however, mastering engineers have started using sophisticated
dynamic compression techniques to achieve higher loudness without
distorting the music (as of 2010, distortion is increasingly being
used in order to achieve even higher loudness).

Unfortunately, this decrease in dynamic range does not leave the music
unharmed.  Current compressed music blasts away your ears and makes
you turn down the volume of your amplifier.  Having lowered the
volume, you'll find that the ``better-sounding'' compressed music
suddenly sounds pretty dull and boring compared to uncompressed music.
In contrast, music with high dynamic range makes you turn up the
volume -- heck, it even sounds better when being broadcast on the
radio!

\chapter{The K-System}
\label{chap:k_system}

The K-System has been devised by mastering engineer Bob Katz in order
to counteract the ongoing loudness race and to help adjusting the
levels of different songs during mastering.  K-System meters are level
meters that do \textbf{not} place the \SI{0}{\dB} mark on top of the
meter.  Instead, \SI{0}{\dB} on K-System meters relates to a reference
loudness.  There are three K-System scales:

\begin{compactitem}
\item K-20 (\SI{0}{\dB} at \SI{-20}{\dBFS}, recommended)
\item K-14 (\SI{0}{\dB} at \SI{-14}{\dBFS})
\item K-12 (\SI{0}{\dB} at \SI{-12}{\dBFS})
\end{compactitem}

Using the K-System is easy.  Just calibrate your monitor system so
that pink noise (\SI{-20}{\dBFS}\,RMS, \SI{20}{\hertz} to
\SI{20}{\kilo\hertz}; \ref{chap:validation} will tell you where to
find a suitable audio file) on one channel yields \SI{83}{\dBSPL} on a
loudness meter placed at your listening position and set to
\emph{C-weighted, slow}.  Then mark the monitor's gain position as
``K-20''.

When your mixes or masters seem to have just the right loudness, they
should now yield \SI{0}{\dB} on a K-20 meter.

In case you want to use the K-14 meter, attenuate the monitor gain by
\SI{6}{\dB} or repeat the above process so that pink noise yields
\SI{77}{\dBSPL}.  For K-12, attenuate the monitor gain by another
\SI{2}{\dB} (pink noise should yield \SI{75}{\dBSPL}).

For more information about the K-System, please see
\href{http://www.digido.com/level-practices-part-2-includes-the-k-system.html}{Bob's
  website} or his great book ``Mastering Audio -- The Art and the
Science''.

\chapter{Installation}
\label{chap:installation}

In order to use the pre-compiled binaries, simply extract the K-Meter
files from the downloaded archive.  For the VST plug-in, you'll then
have to move the extracted files to your plug-in folder
(\path{~/.vst}, \path{C:\Program Files\Steinberg\VstPlugins\} or the
like).

Loading K-Meter may take a few seconds: it checks your computer's
capabilities on start-up so that FFT calculations will run at maximum
speed.  Depending on your computer, this little wait in the beginning
may well result in lower resource usage later.

\section{Windows}

If you move the pre-compiled binaries to another directory, please
make sure to also move the file \path{libfftw3f-3.dll} to this
directory.  Otherwise, you will not be able to use K-Meter.

\chapter{Controls}
\label{chap:controls}

\section{Meter selection}

\begin{wrapfigure}{r}{0pt}
  \includegraphics[scale=\screenshotscale,clip]{include/images/button_meter_selection.png}
\end{wrapfigure}

You can select the different K-System meter scales (\textbf{K-20},
\textbf{K-14} and \textbf{K-12}) by clicking on these radio buttons.

In the rare case you want to use the meter in a mixer's channel strip,
click the \textbf{Normal} button which will place \SI{0}{\dBFS} on
top.  Please note, however, that the \textbf{Normal} state will be
neither saved, nor recalled in your DAW and the standalone version.
This is by design -- the K-System meter has been explicitly designed
to \textbf{not} have \SI{0}{\dBFS} on top!

\section{Averaging method}

\begin{wrapfigure}{r}{0pt}
  \includegraphics[scale=\screenshotscale,clip]{include/images/button_averaging_selection.png}
\end{wrapfigure}

The average level meters can either read unweighted levels
(\textbf{RMS}) or loudness-weighted levels according to
\href{http://www.itu.int/rec/R-REC-BS.1770}{ITU-R BS.1770-1}
(\textbf{ITU-R}).  Click on the corresponding radio button to make
your selection.

According to Bob Katz, the unweighted \textbf{RMS} method has been
designed for stereophonic metering and calibration, while the
loudness-weighted \textbf{ITU-R} method should be used for
channel-summed loudness metering.  To reference a meter, state both
K-System meter scale and averaging method, separated by a slash, such
as ``K-20/ITU''.

\emph{\underline{Note:} K-Meter fully implements Annex 1 of the now
  superseded ITU-R BS.1770-1 standard ('K' frequency weighting, mean
  square calculation and channel-weighted summation), whereas the
  gated loudness measurement specified in ITU-R BS.1770-2 is not yet
  supported.}

\section{Infinite peak hold}

\begin{wrapfigure}{r}{0.14\linewidth}
  \includegraphics[scale=\screenshotscale,clip]{include/images/button_peak_hold_on.png}
  \newline \vspace{-0.9\baselineskip}
  \includegraphics[scale=\screenshotscale,clip]{include/images/button_peak_hold_off.png}
\end{wrapfigure}

Click on this button to toggle between infinite peak hold and
``falling peaks''.  This setting applies to both average and peak
meters.

\section{Show peak level meter}

\begin{wrapfigure}{r}{0.14\linewidth}
  \includegraphics[scale=\screenshotscale,clip]{include/images/button_peak_meter_on.png}
  \newline \vspace{-0.9\baselineskip}
  \includegraphics[scale=\screenshotscale,clip]{include/images/button_peak_meter_off.png}
\end{wrapfigure}

Click on this button to toggle display of the peak level meters.  The
original K-System meter specification demands peak level meters, but
Bob Katz has asked me to hide them by default:

\begin{quotation}
  \emph{``Too many people will try to normalize the peak to full scale
    if they see a peak meter, and that's what we want to avoid.  You
    can still make a K-System meter like the original, but if we meet
    again in \num{15} years I hope that peak metering will be
    outlawed.''}
\end{quotation}

\section{Magnify meters}

\begin{wrapfigure}{r}{0.14\linewidth}
  \includegraphics[scale=\screenshotscale,clip]{include/images/button_expand_meter_on.png}
  \newline \vspace{-0.9\baselineskip}
  \includegraphics[scale=\screenshotscale,clip]{include/images/button_expand_meter_off.png}
\end{wrapfigure}

This button magnifies both average and peak level meters to
\SI{0.1}{\dB} steps.  If switched on, the \SI{0}{\dB} mark is placed
near the centre.

\emph{\underline{Hint:} by selecting different meter scales, you can
  easily magnify the whole range between \SI{-25}{\dBFS} and
  \SI{0}{\dBFS}.}

\section{Mono mode}

\begin{wrapfigure}{r}{0.14\linewidth}
  \includegraphics[scale=\screenshotscale,clip]{include/images/button_mono_on.png}
  \newline \vspace{-0.9\baselineskip}
  \includegraphics[scale=\screenshotscale,clip]{include/images/button_mono_off.png}
\end{wrapfigure}

Click this button to easily check the mono compatibility of your
stereo mix or master.  In \textbf{mono} mode, audio channels will be
down-mixed to mono and the meters will be linked.

In case you insert the plug-in into a mono channel strip,
\textbf{mono} mode will be selected and cannot be toggled.

\section{Reset button}

\begin{wrapfigure}{r}{0.14\linewidth}
  \includegraphics[scale=\screenshotscale,clip]{include/images/button_reset.png}
\end{wrapfigure}

Click on this button to reset all meters, peaks and counters.  You can
also get rid of graphical artifacts, because all meters will be
redrawn as well.

\section{Validation button}
\label{sec:validation_button}

\begin{wrapfigure}{r}{0.14\linewidth}
  \includegraphics[scale=\screenshotscale,clip]{include/images/button_validate_on.png}
  \newline \vspace{-0.9\baselineskip}
  \includegraphics[scale=\screenshotscale,clip]{include/images/button_validate_off.png}
\end{wrapfigure}

Click on this button to open the \textbf{validation window} (see
\ref{chap:validation}) which allows you to play an audio file (WAV,
AIFF or FLAC) through K-Meter and dump internal data.  During
validation, the button will light up and clicking it will stop the
validation.

On Linux, dumped data will be written to \path{stderr}, so just start
the K-Meter standalone or your VST host from the shell and watch the
output coming.  On other systems, have a look at your VST host's log
files (I have successfully used Ableton Live for this).  If that
doesn't work, you might have to start either the K-Meter standalone or
your VST host from a debugger.

As a side note, \textbf{SMA(50)} designates the simple moving average
of 50 values, a neat way to emphasise trends and eliminate short-term
fluctuations.

\newpage %% layout

\section{About button}

\begin{wrapfigure}{r}{0.14\linewidth}
  \includegraphics[scale=\screenshotscale,clip]{include/images/button_about_on.png}
  \newline \vspace{-0.9\baselineskip}
  \includegraphics[scale=\screenshotscale,clip]{include/images/button_about_off.png}
\end{wrapfigure}

Clicking on this button will open the \textbf{about window} where you
will be informed about version number, contributors, copyright and the
GNU General Public License.

\section{Display license}

\begin{wrapfigure}{r}{0.15\linewidth}
  \includegraphics[scale=\screenshotscale,clip]{include/images/button_gpl_on.png}
  \newline \vspace{-0.9\baselineskip}
  \includegraphics[scale=\screenshotscale,clip]{include/images/button_gpl_off.png}
\end{wrapfigure}

This button is located in the \textbf{about window} and does not only
advertise that you are using free software licensed under the
\textbf{GNU General Public License} -- when clicked, it will also open
the license's website in your web browser \dots

\chapter{Meters}
\label{chap:meters}

For \num{5.1} surround sound, K-Meter assumes a channel order of
\emph{L}, \emph{R}, \emph{C}, \emph{LFE}, \emph{Ls} and \emph{Rs}.
Please double-check whether this matches your host's channel order.

\section{K-System meter}

\begin{wrapfigure}{r}{0.14\linewidth}
  \includegraphics[scale=\screenshotscale,clip]{include/images/level_meter_combined.png}
\end{wrapfigure}

The K-System meter consists of an average level meter (graphic on the
right, contiguous lit segments) and an optional peak level meter
(single lit segment on top).  The recommended K-20 meter has
\SI{20}{\dB} of headroom above \SI{0}{\dB}, while the K-14 and K-12
meters have \SI{14}{\dB} and \SI{12}{\dB} of headroom, respectively.

Each level meter is divided into segments of \SI{1}{\dB}, with the
exception of the top \SI{2}{\dB} (segments of \SI{0.5}{\dB}) and the
bottom end (segments of \SI{10}{\dB}).  Magnified level meters are
divided into segments of \SI{0.1}{\dB}.

\emph{\underline{Note:} In ITU-R mode, the average level meter is
  graded in \emph{\si{\LK}} which stands for \emph{Loudness, K
    weighted} and is by all means equivalent to \si{\dB}.}

Recent maximum levels are displayed by white rectangles around the
corresponding meter segments.  Unless ``Infinite peak hold'' is
switched on, maximum levels are held for \SI{10}{\second} and then
start falling with a fall time of \SI{8.67}{\dB\per\second}.

Both stand-alone application and the plug-in only work at sampling
rates between \SI{44.1}{\kilo\hertz} and \SI{192}{\kilo\hertz} and
introduce a latency of \num{1024} samples.  This latency is reported
to your plug-in host so it may compensate for the introduced delay.
Needless to say, the original unfiltered signal is passed to the
outputs.

You can reset all meters by clicking on the ``Reset'' button.

\section{Average level meter}

The average level meter uses an averaging period of \num{1024}
samples.  In \textbf{RMS} mode, this meter exhibits a flat frequency
response between \SI{20}{\hertz} and \SI{20}{\kilo\hertz} (\SI{\pm
  0.01}{\dB}), whereas \textbf{ITU-R} mode implements 'K' frequency
weighting and also sums all channels as specified in
\href{http://www.itu.int/rec/R-REC-BS.1770}{ITU-R BS.1770-1}.

In all modes, the average level meter is band-limited using a
windowed-sinc low-pass filter with a cutoff frequency of
\SI{21.0}{\kilo\hertz}.  On level changes, it takes
\SI{600}{\milli\second} for the meter to reach \SI{99}{\percent} of
the final reading.

\emph{\underline{Note:} Unfortunately, the specifications of ITU-R
  BS.1770-1 clash with those for K-System meters.  I have discussed
  this in depth with Bob Katz and we decided that it makes more sense
  to adhere to ITU-R BS.1770-1 in these cases.}

\emph{Thus, in ITU-R mode sine waves do \emph{not} read the same on
  average and peak level meters.  Moreover, pink noise
  (\SI{-20}{\dBFS}\,RMS, \SI{20}{\hertz} to \SI{20}{\kilo\hertz}) does
  \emph{not} read \SI{0}{\dB} on the K-20 average level meter.  So for
  calibration, please switch K-Meter to RMS mode.}

\section{Peak level meter}

The peak level meter displays the unfiltered peak level and thus
possesses a completely flat frequency response.  It has a rise time of
one sample and a fall time of \SI{8.67}{\dB\per\second}.

\section{Overload counter}

\begin{wrapfigure}{r}{0.23\linewidth}
  \includegraphics[scale=\screenshotscale,clip]{include/images/overload_counter_normal.png}
  \newline \vspace{-0.9\baselineskip}
  \includegraphics[scale=\screenshotscale,clip]{include/images/overload_counter_clipped.png}
\end{wrapfigure}

The overload counter displays the number of samples that have reached
or exceeded digital full scale (to be exact, the counter registers
levels above \SI{-0.001}{\dBFS} to address the granularity of
\num{16}-bit floating-point numbers).  This is a very conservative
approach to estimate overloads -- but I'd rather have an excess
warning than have my audio files clip.

\emph{Please note that this counter does not register inter-sample
  peaks.}

\section{Maximum peak display}

\begin{wrapfigure}{r}{0.23\linewidth}
  \includegraphics[scale=\screenshotscale,clip]{include/images/maximum_peak_normal.png}
  \newline \vspace{-0.9\baselineskip}
  \includegraphics[scale=\screenshotscale,clip]{include/images/maximum_peak_clipped.png}
\end{wrapfigure}

This meter displays the maximum peak level encountered so far in
\si{\dB}.  In case the level exceeds \SI{0.0}{\dBFS} (this can occur
in hosts that do not clip levels to \SI{0.0}{\dBFS}), the meter will
turn red.

\emph{Please note that this display does not register inter-sample
  peaks.}

\section{Phase correlation meter}

\begin{wrapfigure}{r}{0pt}
  \includegraphics[scale=\screenshotscale,clip]{include/images/phase_correlation_meter.png}
\end{wrapfigure}

This meter only works for stereo channels and displays the cross
correlation between left and right channel.  Cross correlation is a
measure of how much two signals are correlated.  Thus, a value of
\num[retainplus]{+1} means that both channels are \emph{in phase},
whereas a value of \num{-1} signals that the channels are completely
\emph{out of phase}.  Please note that the meter's scale is not
linear!

For the non-tech savvy musician: if you find that this meter hits the
red area, you should check the mono-compatibility of your mix.  But
although phase correlation meters often prove helpful, you cannot
always rely on their readout.  The only way to make sure that your
mixes are mono-compatible is to actually listen to them in mono.

That's a universal truth, by the way.  Do not mix by your eyes, mix by
your ears!

\section{Stereo meter}

\begin{wrapfigure}{r}{0pt}
  \includegraphics[scale=\screenshotscale,clip]{include/images/stereo_meter.png}
\end{wrapfigure}

The stereo meter obviously only works for stereo channels and displays
the average stereo position of your mix.  It may indicate a bias to
one stereo channel that you might have overheard due to impaired
hearing, wrong placement of your monitors or similar problems.

But please don't get the false notion that the needle should stay in
the middle all time in order to achieve a good mix.  Quite the
contrary!  As I said, you should not mix by your eyes \dots

\chapter{Validation}
\label{chap:validation}

I have gone to great lengths to ensure that all meters read correctly.
You want to validate for yourself?  Just download and extract the
source code.  The directory \path{validation} contains instructions
and FLAC-compressed wave files.  To validate \textbf{ITU-R} mode,
please download \href{http://www.itu.int/pub/R-REP-BS.2217}{ITU-R
  BS.2217} and follow the instructions (at the time being, the gate
tests should be ignored).  A word of warning: these audio files may
\textbf{damage your ears} and speakers, so please watch your monitor
levels!

\begin{wrapfigure}{r}{0.395\linewidth}
  \includegraphics[scale=\screenshotscale,clip]{include/images/dialog_validation.png}
\end{wrapfigure}

After opening the \textbf{validation window} (see
\ref{sec:validation_button}), click on the ellipsis button (the one
with the dots) to select an audio file for playback through K-Meter.
Please make sure that the sample rates of your host (\textbf{Host SR})
and the audio file match, otherwise the results will not be correct.

Now, select which \textbf{variables} (if any) should be dumped.  You
may also restrict the dumped data to a specific audio
\textbf{channel}.

Finally, click on the \textbf{validate} button to reset all meters and
start playback of the selected audio file.  All audio input will be
discarded during playback and for an additional ten seconds.  To stop
playback early, simply click on the \textbf{validate} button again.

In case you want to calibrate your monitor system, head over to
\href{http://www.digido.com/media/downloads.html}{Bob Katz's download
  section}, get the file labelled \textbf{\SI{-20}{\dBFS} RMS pink
  noise stereo \num{44.1}}, set K-meter to \textbf{RMS} mode and click
on the \textbf{validate} button.  Please ensure that all intermediate
software and hardware mixers are set to the correct levels.

\section{Validation status}

\begin{minipage}{1.0\linewidth}
  \renewcommand{\thempfootnote}{\arabic{mpfootnote}}
  \begin{tabular}{>{\bfseries}llcc}

    &
    \textbf{Readout} &
    \textbf{RMS} &
    \textbf{ITU-R} \\

    Avg level meter &
    meter ballistics &
    \Checkmark{} &
    \Checkmark{} \\

    &
    readings &
    \Checkmark{} &
    --- \\

    &
    frequency response &
    \Checkmark{} &
    \Checkmark{} \\

    &
    pink noise &
    \Checkmark{} &
    --- \\

    &
    ITU-R BS.2217 &
    --- &
    \Checkmark{} \\

    Peak level meter &
    meter ballistics &
    \Checkmark{} &
    \Checkmark{} \\

    &
    readings &
    \Checkmark{} &
    \Checkmark{} \\

    Maximum peak &
    readings &
    \Checkmark{} &
    \Checkmark{} \\

    Overload counter &
    readings &
    \Checkmark{} &
    \Checkmark{} \\

    Phase correlation &
    readings &
    \Checkmark{} &
    \Checkmark{} \\

    Stereo meter &
    readings &
    \Checkmark{} &
    \Checkmark{} \\

  \end{tabular}
\end{minipage}

\newpage %% layout

\section{Frequency and phase response}

Frequency and phase response have been determined at a sample rate of
\SI{192}{\kilo\hertz} using
\href{http://www.savioursofsoul.de/Christian/programs/measurement-programs/}{VST
  Plugin Analyser}.

\textbf{Frequency response of complete effect path (\SI{5}{\hertz} to
  \SI{96}{\kilo\hertz}, \SI{0}{\dB} \SI{\pm 0.1}{\dB}):}

\begin{center}
  \includegraphics[scale=0.65,clip]{include/images/fft_192khz-freq-fx_path.png}
\end{center}

\textbf{Phase response of complete effect path (\SI{5}{\hertz} to
  \SI{96}{\kilo\hertz}, \SI{0}{\degree}\,\SI{\pm 0.1}{\degree}):}

\begin{center}
  \includegraphics[scale=0.65,clip]{include/images/fft_192khz-phase-fx_path.png}
\end{center}

\newpage %% layout

\textbf{Frequency response of band-limited RMS detection stage
  (\SI{5}{\hertz} to \SI{96}{\kilo\hertz}, \SI{-140}{\dB} to
  \SI{5}{\dB}):}

\begin{center}
  \includegraphics[scale=0.65,clip]{include/images/fft_192khz-freq-rms.png}
\end{center}

\textbf{Phase response of band-limited RMS detection stage
  (\SI{5}{\hertz} to \SI{96}{\kilo\hertz}, \SI{-180}{\degree} to
  \SI[addsign]{+180}{\degree}):}

\begin{center}
  \includegraphics[scale=0.65,clip]{include/images/fft_192khz-phase-rms.png}
\end{center}
\newpage %% layout

\textbf{Frequency response of band-limited ITU-R BS.1770-1 detection stage
  (\SI{5}{\hertz} to \SI{96}{\kilo\hertz}, \SI{-140}{\dB} to
  \SI{5}{\dB}):}

\begin{center}
  \includegraphics[scale=0.65,clip]{include/images/fft_192khz-freq-itu_r.png}
\end{center}

\textbf{Phase response of band-limited ITU-R BS.1770-1 detection stage
  (\SI{5}{\hertz} to \SI{96}{\kilo\hertz}, \SI{-180}{\degree} to
  \SI[addsign]{+180}{\degree}):}

\begin{center}
  \includegraphics[scale=0.65,clip]{include/images/fft_192khz-phase-itu_r.png}
\end{center}

\newpage %% layout

\textbf{Frequency response of band-limited RMS detection stage
  (\SI{5}{\hertz} to \SI{96}{\kilo\hertz}, \SI{0}{\dB} \SI{\pm
    1}{\dB}):}

\begin{center}
  \includegraphics[scale=0.65,clip]{include/images/fft_192khz-freq_zoomed-rms.png}
\end{center}

\textbf{Frequency response of band-limited ITU-R BS.1770-1 detection stage
  (\SI{5}{\hertz} to \SI{96}{\kilo\hertz}, \SI{0}{\dB} \SI{-6}{\dB} to
  \SI{4}{\dB}):}

\begin{center}
  \includegraphics[scale=0.65,clip]{include/images/fft_192khz-freq_zoomed-itu_r.png}
\end{center}

\chapter{Help needed}
\label{chap:help_needed}

As K-Meter was coded using cross-platform code, it should be easy to
compile versions for Windows (\num{64} bit) and Mac OS X.  I just
don’t have the adequate systems and compilers.

In case you want to help, please see the next chapter for an email
address.  You’ll need sufficient experience in coding, compiling and
debugging, though, so no beginners please!

\chapter{Final words}
\label{chap:final_words}

I want to express my gratitude to \textbf{Bob Katz} for kindly
answering all of my questions regarding the K-System meter and
checking this document for technical errors.  I'd further like to
thank \textbf{bram@smartelectronix} for his code to calculate
logarithmic rise and fall times, and \textbf{Raiden} for working out
the ITU-R BS.1770-1 filter specifications.  I must also thank the
\textbf{users of K-Meter} for sending kind words and bug reports,
offering suggestions and testing things.  Finally, I want to thank the
\textbf{open source community} for making all of this possible.

Although coding K-Meter has been a lot of fun, it has also been a lot
of work.  So if you like K-Meter, why not send me a short email and
tell me so?  Write a few words about yourself, send suggestions for
future updates or volunteer to create a nice theme -- do whatever you
like!

Here is my email address (please remove ``\texttt{-nospam}''):

\begin{center}
  \texttt{"Martin Zuther" <code-nospam@mzuther.de>}
\end{center}

Thanks for using free software.  I hope you'll enjoy it!

\emph{VST is a trademark of Steinberg Media Technologies GmbH.  ASIO
  is a trademark and software of Steinberg Media Technologies GmbH.}

\appendix

\chapter{How to build K-Meter}
\label{chap:build_kmeter}

\section{Preparing GNU/Linux}

To build K-Meter yourself, I recommend setting up a \texttt{chroot}
environment.  This is fast and easy to do on Debian-based systems and
might save you a \textbf{lot} of trouble.  At the time of writing, I'm
using Linux Mint 13 (Maya), but the procedure should be similar on
your distribution of choice.  If you aim at generic \num{64}-bit
compilation, simply change \texttt{i386} to \texttt{amd64}.

To install the necessary packages and install the \texttt{chroot} base
system, execute the following statements (please change
\path{http://ftp.de.debian.org/debian/} to a
\href{http://www.debian.org/mirror/list}{mirror} close to you):

\begin{verbatim}
  sudo apt-get install debootstrap schroot

  sudo mkdir -p /srv/chroot/squeeze_i386
  sudo debootstrap --variant=buildd \
    --arch i386 squeeze \
    /srv/chroot/squeeze_i386 \
    http://ftp.de.debian.org/debian/
\end{verbatim}

Running \path{debootstrap} will take some time.  Meanwhile, add the
following lines to \path{/etc/schroot/schroot.conf} (make sure you
remove all preceding white space so that each line begins in the first
column):

\begin{verbatim}
  [squeeze-i386]
  description=Debian 6 (Squeeze, i386)
  directory=/srv/chroot/squeeze_i386
  personality=linux
  root-users=username
  type=directory
  users=username,another_user
\end{verbatim}

Please make the necessary changes to \texttt{username}.  You may also
add additional users, like \texttt{another\_user}.  In case you are
setting up a \num{32}-bit \texttt{chroot} environment on a
\num{64}-bit system, you'll also have to change \texttt{linux} to
\texttt{linux32}.

When \path{debootstrap} is done, log in as superuser:

\begin{verbatim}
  schroot -c squeeze-i386 -u root
\end{verbatim}

to install a few packages.  The packages \path{less} and \path{vim}
are optional, but might come in handy:

\begin{verbatim}
  apt-get update
  apt-get -y install bash-completion libasound2-dev \
    libjack-jackd2-dev mesa-common-dev xorg-dev
  apt-get -y install less vim
  apt-get clean
\end{verbatim}

If you like \path{bash} completion, you might also want to open the
file \path{/etc/bash.bashrc} and unquote these lines:

\begin{verbatim}
  # enable bash completion in interactive shells
  [two more lines...]
  fi
\end{verbatim}

Finally, log out and log in as normal user:

\begin{verbatim}
  schroot -c squeeze-i386
\end{verbatim}

Congratulations -- after you have installed the dependencies (see
below), you are ready to build K-Meter!

\section{Dependencies}

\subsection{premake4}

\begin{tabbing}
  \hspace*{6em}\=\=\kill

  Importance:  \> required \\
  Version:     \> 4.3 \\
  License:     \> BSD \\
  Homepage:    \> \href{http://industriousone.com/premake}{industriousone.com/premake}
\end{tabbing}

\subsubsection{Installation}

Place the binary somewhere in your \path{PATH}.  Depending on your
platform, you should run \emph{premake} using the scripts
\path{build/run_premake.sh} or \path{build/run_premake.bat}.

\newpage %% layout

\subsection{Fastest Fourier Transform in the West}

\begin{tabbing}
  \hspace*{6em}\=\=\kill

  Importance:  \> required \\
  Version:     \> 3.3.2 \\
  License:     \> GPL v2 \\
  Homepage:    \> \href{http://www.fftw.org/}{www.fftw.org}
\end{tabbing}

\subsubsection{Installation on GNU/Linux}

Extract the archive into the directory \path{libraries/fftw3}, change
into this directory and run:

\begin{verbatim}
  ./configure --enable-float --with-pic
  make
  mkdir -p bin/i386/
  mv .libs/* bin/i386/
\end{verbatim}

\subsubsection{Installation on Microsoft Windows}

Extract the source code archive into the directory
\path{libraries/fftw3} and the archive containing the pre-compiled
binaries into the directory \path{libraries/fftw3/bin}.

Please note that in order to run K-Meter on Windows, the library
\path{libfftw3f-3.dll} \textbf{must} be located in the same directory
as the standalone or plug-in.  To make things a little easier for you,
I have already placed \path{libfftw3f-3.dll} in the directories
\path{bin} and \path{bin/final}.

\subsection{JUCE library}

\begin{tabbing}
  \hspace*{6em}\=\=\kill

  Importance:  \> required \\
  Version:     \> 1.53 \\
  License:     \> GPL v2 \\
  Homepage:    \> \href{http://www.rawmaterialsoftware.com/juce.php}{www.rawmaterialsoftware.com/juce.php}
\end{tabbing}

\subsubsection{Installation}

Extract the archive into the directory \path{libraries/juce}.

\subsection{Virtual Studio Technology SDK}

\begin{tabbing}
  \hspace*{6em}\=\=\kill

  Importance:  \> optional \\
  Version:     \> 2.4 \\
  License:     \> proprietary \\
  Homepage:    \> \href{http://ygrabit.steinberg.de/}{ygrabit.steinberg.de}
\end{tabbing}

\subsubsection{Installation}

Just extract the archive into the directory
\path{libraries/vstsdk2.4}.

\subsection{Audio Streaming Input Output SDK}

\begin{tabbing}
  \hspace*{6em}\=\=\kill

  Importance:  \> optional \\
  Version:     \> 2.2 \\
  License:     \> proprietary \\
  Homepage:    \> \href{http://ygrabit.steinberg.de/}{ygrabit.steinberg.de}
\end{tabbing}

\subsubsection{Installation}

Simply extract the archive into the directory
\path{libraries/asiosdk2.2}.

\subsection{Artistic Style}

\begin{tabbing}
  \hspace*{6em}\=\=\kill

  Importance:  \> optional \\
  Version:     \> 2.01 \\
  License:     \> LGPL v3 \\
  Homepage:    \> \href{http://astyle.sourceforge.net/}{astyle.sourceforge.net}
\end{tabbing}

This application formats the code so it looks more beautiful and
consistent.  Thus, you only have to install it if you plan to help me
with coding K-Meter.

\subsubsection{Installation}

Place the binary somewhere in your \path{PATH}.  Depending on your
platform, you should run \emph{astyle} using the scripts
\path{src/format_code.sh} or \path{src/format_code.bat}.

\section{Building on GNU/Linux}

After preparing the dependencies, start your \texttt{chroot}
environment, change into the directory \path{build} and execute

\begin{verbatim}
  ./run_premake.sh
  make config=CFG TARGET
\end{verbatim}

where \application{CFG} is one of \application{debug32},
\application{debug64}, \application{release32} and
\application{release64}, and \application{TARGET} is one of
\application{linux\_standalone\_stereo},
\application{linux\_standalone\_surround},
\application{linux\_vst\_stereo} and
\application{linux\_vst\_surround}.

The compiled binaries will end up in the directory \path{bin}.

\section{Building on Microsoft Windows}

After preparing the dependencies, change into the directory
\path{build} and execute

\begin{verbatim}
  ./run_premake.bat
\end{verbatim}

Then change into the directory \path{build/windows/vs20xx}, open the
project file with the corresponding version of Visual C++ and build
the project.

The compiled binaries will end up in the directory \path{bin}.

\input{include/gpl_v3.tex}

\end{document}


%%% Local Variables:
%%% mode: latex
%%% mode: outline-minor
%%% TeX-command-default: "Rubber"
%%% TeX-PDF-mode: t
%%% ispell-local-dictionary: "british"
%%% End:

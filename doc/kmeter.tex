\documentclass[paper=a5,pagesize=auto,twoside=false,fontsize=12pt,DIV=classic,BCOR=0mm,headinclude=true,footinclude=false]{scrartcl} % final

%% font selection
\usepackage{utopia} % final
\usepackage{avant} % final

%% redefine headers
\usepackage[headsepline]{scrpage2}
\pagestyle{scrheadings}

\setkomafont{pageheadfoot}{%
  \normalfont\itshape
}

\automark[section]{chapter}
\renewcommand{\chaptermark}[1]{\markboth{#1}{#1}}
\renewcommand{\sectionmark}[1]{\markright{#1}}

%% additional KOMA script options
\KOMAoptions{parskip=half,headsepline=true,numbers=noendperiod,captions=tableheading}
\recalctypearea{} % recalculate type area after selecting fonts

\usepackage[british]{babel}
\usepackage{graphicx}
\usepackage{ifpdf}
\usepackage[utf8x]{inputenc}
\usepackage{paralist}
\usepackage{ucs}
\usepackage[hyphens]{url}

% Make sure "hyperref" comes last of your loaded packages, to give it
% a fighting chance of not being over-written, since its job is to
% redefine many LATEX commands.
\ifpdf
	\usepackage[unicode,pdftex]{hyperref}
\else
	\usepackage[unicode,hypertex]{hyperref}
\fi

\hypersetup{ % final
  unicode=true,                % non-Latin characters in Acrobat’s bookmarks
  pdftitle=K-Meter,            % title
  pdfauthor=Martin Zuther,     % author
  pdfsubject=K-Meter,          % subject of the document
  pdfcreator=Martin Zuther,    % creator of the document
  colorlinks=true,             % false: boxed links; true: colored links
  linkcolor=blue,              % color of internal links
  citecolor=blue,              % color of links to bibliography
  filecolor=blue,              % color of file links
  urlcolor=blue                % color of external links
}

%% KOMA script captions
\addtokomafont{caption}{\small}
\addtokomafont{captionlabel}{\bfseries}
\setcapindent{0em}

%% taken from package "l2tabu"
\tolerance 1414
\hbadness 1414
\emergencystretch 1.5em
\hfuzz 0.3pt
\widowpenalty=10000
\vfuzz
\hfuzz
\raggedbottom

%% package "paralist"
\setdefaultitem{\textbullet}{$\circ$}{}{}

%% package "url"
\urlstyle{rm}
\input{include/hyphenation.sty}

\title{K-Meter}
\author{Martin Zuther}

\begin{document}

\title{K-Meter}

\subtitle{
  \normalsize{\textrm{\textmd{
        \vfill
        Free implementation of a K-System meter \\
        according to Bob Katz' specifications
        \vfill
        \vspace{1.5em}
        \begin{figure}[H]
          \centering{}
          \includegraphics[scale=0.275,clip]{include/images/kmeter.png}
        \end{figure}
        \vfill
      }}}
}

\author{\normalsize\copyright{} 2010-2011
  \href{http://www.mzuther.de/}{Martin Zuther}}

\date{\normalsize \emph{Last edited on \today}}

\maketitle

\tableofcontents

\clearpage  % layout

\chapter{The loudness race}

When comparing two similar pieces of music, the louder one is
perceived as sounding better (although this is only true for short
periods of time).  Accordingly, the loudness of music productions has
continuously grown during the last decades.

As maximum levels of records, tapes and digital media have a natural
limit, however, mastering engineers have started using sophisticated
dynamic compression techniques to achieve higher loudness without
distorting the music (as of 2010, distortion is increasingly being
used in order to achieve even higher loudness).

Unfortunately, this decrease in dynamic range does not leave the music
unharmed.  Current compressed music blasts away your ears and makes
you turn down the volume of your amplifier.  Having lowered the
volume, you'll find that the ``better-sounding'' compressed music
suddenly sounds pretty dull and boring compared to uncompressed music.
In contrast, music with high dynamic range makes you turn up the
volume -- heck, it even sounds better when being broadcast on the
radio!

\chapter{The K-System}

The K-System has been devised by mastering engineer Bob Katz in order
to counteract the ongoing loudness race and to help adjusting the
levels of different songs during mastering.  K-System meters are level
meters that do \textbf{not} place the 0\,dB mark on top of the meter.
Instead, 0\,dB on K-System meters relates to a reference loudness.
There are three K-System scales:

\begin{compactitem}
\item K-20 (0\,dB at -20\,dBFS, recommended)
\item K-14 (0\,dB at -14\,dBFS)
\item K-12 (0\,dB at -12\,dBFS)
\end{compactitem}

Using the K-System is easy.  Just calibrate your monitor system so
that pink noise (-20\,dBFS RMS, 20\,Hz to 20\,kHz; see the K-Meter
source code for a FLAC-compressed wave file) on one channel yields
83\,dB\,SPL on a loudness meter set to \emph{C-weighted, slow}.  Then
mark the monitor's gain position as ``K-20''.

When your mixes or masters seem to have just the right loudness, they
should now yield 0\,dB on a K-20 meter.

In case you want to use the K-14 meter, attenuate the monitor gain by
6\,dB or repeat the above process so that pink noise yields
77\,dB\,SPL.  For K-12, attenuate the monitor gain by another 2\,dB
(pink noise should yield 75\,dB\,SPL).

For more information about the K-System, please see
\href{http://www.digido.com/level-practices-part-2-includes-the-k-system.html}{Bob's
  website} or his great book ``Mastering Audio -- The Art and the
Science''.

\chapter{Installation}

In order to use the pre-compiled binaries, simply extract the K-Meter
files from the downloaded archive.  For the VST plug-in, you'll then
have to move the extracted files to your plug-in folder
(\path{~/.vst}, \path{C:\Program Files\Steinberg\VstPlugins\} or the
  like).

\section{Windows}

If you move the pre-compiled binaries to another directory, please
make sure to also move the file \path{libfftw3f-3.dll} to this
directory.  Otherwise, you will not be able to execute the binaries
anymore.

\chapter{Controls}

\section{Meter selection}

\begin{wrapfigure}{r}{0pt}
  \includegraphics[scale=\screenshotscale,clip]{include/images/button_meter_selection.png}
\end{wrapfigure}

You can select the different K-System meter scales (\textbf{K-20}, \textbf{K-14} and
\textbf{K-12}) by clicking on these radio buttons.

In the rare case you want to use the meter in a mixer's channel strip,
click the \textbf{Normal} button which will place 0\,dBFS on top.
Please note, however, that the \textbf{Normal} state will be neither
saved, nor recalled in your DAW and the standalone version.  This is
by design -- the K-System meter has been explicitly designed to
\textbf{not} have 0\,dBFS on top!

\section{Infinite peak hold}

\begin{wrapfigure}{r}{0.14\linewidth}
  \includegraphics[scale=\screenshotscale,clip]{include/images/button_peak_hold_on.png}
  \newline \vspace{-0.9\baselineskip}
  \includegraphics[scale=\screenshotscale,clip]{include/images/button_peak_hold_off.png}
\end{wrapfigure}

Click on this button to toggle between infinite peak hold and
``falling peaks''.  This setting applies to both average and peak
meters.

\section{Show peak meter}

\begin{wrapfigure}{r}{0.14\linewidth}
  \includegraphics[scale=\screenshotscale,clip]{include/images/button_peak_meter_on.png}
  \newline \vspace{-0.9\baselineskip}
  \includegraphics[scale=\screenshotscale,clip]{include/images/button_peak_meter_off.png}
\end{wrapfigure}

Click on this button to toggle display of the peak meters.  The
original K-System meter specification demands peak meters, but Bob
Katz has asked me to hide them by default:

\begin{quotation}
  \emph{``Too many people will try to normalize the peak to full scale
    if they see a peak meter, and that's what we want to avoid.  You
    can still make a K-System meter like the original, but if we meet
    again in 15 years I hope that peak metering will be outlawed.''}
\end{quotation}

\section{Magnify meters}

\begin{wrapfigure}{r}{0.14\linewidth}
  \includegraphics[scale=\screenshotscale,clip]{include/images/button_expand_meter_on.png}
  \newline \vspace{-0.9\baselineskip}
  \includegraphics[scale=\screenshotscale,clip]{include/images/button_expand_meter_off.png}
\end{wrapfigure}

This button magnifies both average and peak meters to 0.1\,dB steps.
If switched on, the 0\,dB mark is placed near the centre.

\emph{\underline{Hint:} by selecting different meter scales, you can
  easily magnify the whole range between -25\,dBFS and 0\,dBFS.}

\section{Mono mode}

\begin{wrapfigure}{r}{0.14\linewidth}
  \includegraphics[scale=\screenshotscale,clip]{include/images/button_mono_on.png}
  \newline \vspace{-0.9\baselineskip}
  \includegraphics[scale=\screenshotscale,clip]{include/images/button_mono_off.png}
\end{wrapfigure}

Click this button to easily check the mono compatibility of your
stereo mix or master.  In \textbf{mono} mode, audio channels will be
down-mixed to mono and the meters will be linked.

In case you insert the plug-in into a mono channel strip,
\textbf{mono} mode will be selected and cannot be toggled.

\section{Reset button}

\begin{wrapfigure}{r}{0.14\linewidth}
  \includegraphics[scale=\screenshotscale,clip]{include/images/button_reset.png}
\end{wrapfigure}

Click on this button to reset all meters, peaks and counters.  You can
also get rid of graphical artifacts, because all meters will be
redrawn as well.

\section{About window}

\begin{wrapfigure}{r}{0.14\linewidth}
  \includegraphics[scale=\screenshotscale,clip]{include/images/button_about_on.png}
  \newline \vspace{-0.9\baselineskip}
  \includegraphics[scale=\screenshotscale,clip]{include/images/button_about_off.png}
\end{wrapfigure}

Clicking on this button will open the \textbf{about} window where you
will be informed about version number, contributors, copyright and the
GNU General Public License.

\section{Display license}

\begin{wrapfigure}{r}{0.15\linewidth}
  \includegraphics[scale=\screenshotscale,clip]{include/images/button_gpl_on.png}
  \newline \vspace{-0.9\baselineskip}
  \includegraphics[scale=\screenshotscale,clip]{include/images/button_gpl_off.png}
\end{wrapfigure}

This button is located in the \textbf{about window} and does not only
advertise that you are using free software licensed under the
\textbf{GNU General Public License} -- when clicked, it will also open
the license's website in your web browser \dots

\chapter{Meters}

\section{K-System meter}

\begin{wrapfigure}{r}{0.14\linewidth}
  \includegraphics[scale=\screenshotscale,clip]{include/images/level_meter_combined.png}
\end{wrapfigure}

The K-System meter consists of an average level meter (graphic on the
right, wide bars) and an optional peak level meter (thin bars).  The
recommended K-20 meter has 20\,dB of headroom above 0\,dB, while the
K-14 and K-12 meters have 14\,dB and 12\,dB of headroom, respectively.

Each level meter is divided into segments of 1\,dB, with the exception
of the top 2\,dB (segments of 0.5\,dB) and the bottom end (segments of
10\,dB).  When magnified, level meters are divided into segments of
0.1\,dB.

Recent maximum level is displayed by a white rectangle around the
corresponding meter segment.  Unless ``Infinite peak hold'' is
switched on, maximum levels are held for 10 seconds and then start
falling with a fall time of 26\,dB per 3 seconds.

Both stand-alone application and the plug-in only work at sampling
rates between 44.1\,kHz and 192\,kHz and introduce a latency of 1024
samples.  This latency is reported to your plug-in host so it may
compensate for the introduced delay.  Needless to say, the original
unfiltered signal is passed to the outputs.

You can reset all meters by clicking on the ``Reset'' button.

\section{Average meter}

The average meter is an RMS meter with an averaging period of 1024
samples.  It exhibits a flat frequency response between 20\,Hz and
20\,kHz ($\pm$0.01\,dB) and is band-limited using a windowed-sinc
low-pass filter with a cutoff frequency of 21.0\,kHz.  On level
changes, it takes 600 milliseconds for the meter to reach 99\,\% of
the final reading.

\section{Peak meter}

The peak meter displays the unfiltered peak level and thus possesses a
completely flat frequency response.  It has a rise time of one sample
and a fall time of 26\,dB per 3 seconds.

\section{Overload counter}

\begin{wrapfigure}{r}{0.23\linewidth}
  \includegraphics[scale=\screenshotscale,clip]{include/images/overload_counter_normal.png}
  \newline \vspace{-0.9\baselineskip}
  \includegraphics[scale=\screenshotscale,clip]{include/images/overload_counter_clipped.png}
\end{wrapfigure}

The overload counter displays the number of samples that have reached
or exceeded digital full scale (0\,dBFS).  This is a conservative
approach to estimate overloads -- but I'd rather have an excess
warning than have my audio files clip.

Please note that this meter does not search for inter-sample peaks.

\section{Maximum peak display}

\begin{wrapfigure}{r}{0.23\linewidth}
  \includegraphics[scale=\screenshotscale,clip]{include/images/maximum_peak_normal.png}
  \newline \vspace{-0.9\baselineskip}
  \includegraphics[scale=\screenshotscale,clip]{include/images/maximum_peak_clipped.png}
\end{wrapfigure}

This meter displays the maximum peak level encountered so far.  If
this level is above 0\,dBFS (this can occur in plug-in hosts that do
not clip the output level of plug-ins to 0\,dBFS), the meter will turn
red.

\section{Phase correlation meter}

\begin{wrapfigure}{r}{0pt}
  \includegraphics[scale=\screenshotscale,clip]{include/images/phase_correlation_meter.png}
\end{wrapfigure}

This meter only works for stereo channels and displays the cross
correlation between left and right channel.  Cross correlation is a
measure of how much two signals are correlated.  Thus, a value of +1
means that both channels are \emph{in phase}, whereas a value of -1
signals that the channels are completely \emph{out of phase}.  Please
note that the meter's scale is not linear!

For the non-tech savvy musician: if you find that this meter hits the
red area, you should check the mono-compatibility of your mix.  But
although phase correlation meters often prove helpful, you cannot
always rely on their readout.  The only way to make sure that your
mixes are mono-compatible is to actually listen to them in mono.

That's a universal truth, by the way.  Do not mix by your eyes, mix by
your ears!

\section{Stereo meter}

\begin{wrapfigure}{r}{0pt}
  \includegraphics[scale=\screenshotscale,clip]{include/images/stereo_meter.png}
\end{wrapfigure}

The stereo meter obviously only works for stereo channels and displays
the average stereo position of your mix.  It may indicate a bias to
one stereo channel that you might have overheard due to impaired
hearing, wrong placement of your monitors or similar problems.

But please don't get the false notion that the needle should stay in
the middle all time in order to achieve a good mix!  Again: do not mix
by your eyes \dots

\chapter{Validation}

I have gone to great lengths to ensure that all meters of the stereo
version read correctly (the surround version shares the same code and
should follow suit).  You want to validate for yourself?  The
directory \path{validation} of the source code contains instructions
and FLAC-compressed wave files.

\section{Validation status}

\begin{minipage}{1.0\linewidth}
  \renewcommand{\thempfootnote}{\arabic{mpfootnote}}
  \begin{tabular}{>{\bfseries}lll}

    average level meter &
    readings &
    valid \\

    &
    frequency response &
    valid \\

    &
    meter ballistics &
    valid \\

    peak level meter &
    readings &
    ---\footnote{deviates from the correct value by a small
      amount, probably due to dithering in the used hosts
      \label{fn:validation_dithering}} \\

    &
    meter ballistics &
    valid \\

    overload counter &
    readings &
    ---\footnote{doesn't validate correctly, probably due to dithering
      in the used hosts} \\

    maximum peak display &
    readings &
    ---\footref{fn:validation_dithering}\\

    phase correlation meter &
    readings &
    valid \\

    stereo meter &
    readings &
    valid \\

  \end{tabular}
\end{minipage}

\section{Frequency and phase response}

The VST plug-in's frequency and phase response has been determined at
a sample rate of 192\,kHz using Christian-W. Budde's
\href{http://www.savioursofsoul.de/Christian/programs/measurement-programs/}{VST
  Plugin Analyser}.

\textbf{Frequency response of complete effect path (5\,Hz to 96\,kHz,
  0\,dB $\pm$0.01\,dB):}

\begin{center}
  \includegraphics[scale=0.65,clip]{include/images/fft_192khz_freq-fx_path.png}
\end{center}

\textbf{Phase response of complete effect path (5\,Hz to 96\,kHz,
  0\textdegree{} $\pm$0.1\textdegree{}):}

\begin{center}
  \includegraphics[scale=0.65,clip]{include/images/fft_192khz_phase-fx_path.png}
\end{center}

\textbf{Frequency response of band-limited RMS detection stage (5\,Hz
  to 96\,kHz, -140\,dB to 0\,dB):}

\begin{center}
  \includegraphics[scale=0.65,clip]{include/images/fft_192khz_freq-rms.png}
\end{center}

\textbf{Phase response of band-limited RMS detection stage (5\,Hz to
  96\,kHz, -180\textdegree{} to 180\textdegree{}):}

\begin{center}
  \includegraphics[scale=0.65,clip]{include/images/fft_192khz_phase-rms.png}
\end{center}

\newpage %% layout

\textbf{Frequency response of band-limited RMS detection stage (5\,Hz
  to 20\,kHz, 0\,dB $\pm$0.01\,dB):}

\begin{center}
  \includegraphics[scale=0.65,clip]{include/images/fft_192khz_freq-rms_zoomed.png}
\end{center}

\chapter{Help needed}

As K-Meter was coded using cross-platform code, it should be easy to
compile versions for Windows (64 bit) and Mac OS X.  I just don’t have
the adequate systems and compilers.

In case you want to help, please see the next chapter for an email
address.  You’ll need sufficient experience in coding, compiling and
debugging, though, so no beginners please!

\chapter{Final words}

I'd like to thank Bob Katz for kindly answering all of my questions
regarding the K-System meter.  Moreover, thanks are due to
bram@smartelectronix for his code to calculate logarithmic rise and
fall times, and to the users of K-Meter for their suggestions and bug
reports.

Although coding K-Meter has been a lot of fun, it has also been a lot
of work.  So if you like K-Meter, why not send me a short email and
tell me so?  Write a few words about yourself, send suggestions for
future updates or volunteer to create a nice theme -- do whatever you
like!

Here is my email address (please remove ``\texttt{-nospam}''):

\begin{center}
  \texttt{Martin Zuther <code-nospam@mzuther.de>}
\end{center}

Thanks for using free software.  I hope you'll enjoy it!

\appendix

\chapter{How to build K-Meter}

\section{Preparing GNU/Linux}

To build K-Meter yourself, I recommend setting up a \texttt{chroot}
environment.  This is fast and easy to do on Debian-based systems and
might save you a \textbf{lot} of trouble.  At the time of writing, I'm
using Kubuntu 10.10 (\texttt{maverick}), but the procedure should be
similar on your distribution of choice.  If you aim at generic 64-bit
compilation, simply change \texttt{i386} to \texttt{amd64}.

To install the necessary packages and install the \texttt{chroot} base
system, execute the following statements (please change
\url{http://archive.ubuntu.com/ubuntu/} to a
\href{http://launchpad.net/ubuntu/+archivemirrors}{mirror} close to
you):

\begin{verbatim}
  sudo apt-get install debootstrap schroot

  sudo mkdir -p /srv/chroot/maverick_i386
  sudo debootstrap --variant=buildd \
    --arch i386 maverick \
    /srv/chroot/maverick_i386 \
    http://archive.ubuntu.com/ubuntu/
\end{verbatim}

Running \path{debootstrap} will take some time.  Meanwhile, add the
following lines to \path{/etc/schroot/schroot.conf} (make sure you
remove all preceding white space so that each line begins in the first
column):

\begin{verbatim}
  [maverick-i386]
  description=Ubuntu 10.10 Maverick Meerkat (i386)
  directory=/srv/chroot/maverick_i386
  personality=linux
  root-users=username
  type=directory
  users=username,another_user
\end{verbatim}

Please make the necessary changes to \texttt{username}.  You may also
add additional users, like \texttt{another\_user}.  In case you are
setting up a 32-bit \texttt{chroot} environment on a 64-bit system,
you'll also have to change \texttt{linux} to \texttt{linux32}.

When \path{debootstrap} is done, log in as superuser:

\begin{verbatim}
  schroot -c maverick-i386 -u root
\end{verbatim}

to install a few packages:

\begin{verbatim}
  apt-get update
  apt-get install libasound2-dev mesa-common-dev \
    xorg-dev language-pack-de ubuntu-minimal
  apt-get clean
\end{verbatim}

Finally, log out and log in with your user name:

\begin{verbatim}
  schroot -c maverick-i386 -u username
\end{verbatim}

Congratulations -- after you have installed the dependencies (see
below), you are ready to build K-Meter!

\section{Dependencies}

\subsection{premake4}

\begin{tabbing}
  \hspace*{6em}\=\=\kill

  Importance:  \> required \\
  Version:     \> 4.3 \\
  License:     \> BSD \\
  Homepage:    \> \href{http://industriousone.com/premake}{industriousone.com/premake}
\end{tabbing}

\subsubsection{Installation}

Place the binary somewhere in your \path{PATH}.

\subsection{Fastest Fourier Transform in the West}

\begin{tabbing}
  \hspace*{6em}\=\=\kill

  Importance:  \> required \\
  Version:     \> 3.2.2 \\
  License:     \> GPL v2 \\
  Homepage:    \> \href{http://www.fftw.org/}{www.fftw.org}
\end{tabbing}

\subsubsection{Installation on GNU/Linux}

Extract the archive into the directory \path{libraries/fftw3}, change
into this directory and run:

\begin{verbatim}
  ./configure --enable-float
  make
  mkdir -p bin/i386/
  mv .libs/* bin/i386/
\end{verbatim}

\subsubsection{Installation on Microsoft Windows}

Extract the source code archive into the directory
\path{libraries/fftw3} and the archive containing the pre-compiled
binaries into the directory \path{libraries/fftw3/bin}.

Please note that in order to run K-Meter on Windows, the library
\path{libfftw3f-3.dll} \textbf{must} be located in the same directory
as the standalone or plug-in.  To make things a little easier for you,
I have already placed \path{libfftw3f-3.dll} in the directory
\path{bin}.

\subsection{JUCE library}

\begin{tabbing}
  \hspace*{6em}\=\=\kill

  Importance:  \> required \\
  Version:     \> 1.51 \\
  License:     \> GPL v2 \\
  Homepage:    \> \href{http://www.rawmaterialsoftware.com/juce.php}{www.rawmaterialsoftware.com/juce.php}
\end{tabbing}

\subsubsection{Installation}

Extract the archive into the directory \path{libraries/juce}.

\subsection{Virtual Studio Technology SDK (VST)}

\begin{tabbing}
  \hspace*{6em}\=\=\kill

  Importance:  \> optional \\
  Version:     \> 2.4 \\
  License:     \> proprietary \\
  Homepage:    \> \href{http://ygrabit.steinberg.de/}{ygrabit.steinberg.de}
\end{tabbing}

\subsubsection{Installation}

Just extract the archive into the directory
\path{libraries/vstsdk2.4}.

\subsection{Artistic Style}

\begin{tabbing}
  \hspace*{6em}\=\=\kill

  Importance:  \> optional \\
  Version:     \> 1.24 \\
  License:     \> LGPL v3 \\
  Homepage:    \> \href{http://astyle.sourceforge.net/}{astyle.sourceforge.net}
\end{tabbing}

This application formats the code so it looks more beautiful and
consistent.  Thus, you only have to install it if you plan to help me
with coding K-Meter.

\subsubsection{Installation}

Place the binary somewhere in your \path{PATH}.  Depending on your
platform, you should run \emph{astyle} using the scripts
\path{src/format_code.sh} or \path{src/format_code.bat}.

\section{Building on GNU/Linux}

After preparing the dependencies, start your \texttt{chroot}
environment, change into the directory \path{build} and execute

\begin{verbatim}
  ./run_premake.sh
  make config=CFG TARGET
\end{verbatim}

where \application{CFG} is one of \application{debug32},
\application{debug64}, \application{release32} and
\application{release64}, and \application{TARGET} is one of
\application{linux\_standalone\_stereo},
\application{linux\_standalone\_surround},
\application{linux\_vst\_stereo} and
\application{linux\_vst\_surround}.

The compiled binaries will end up in the directory \path{bin}.

\section{Building on Microsoft Windows}

After preparing the dependencies, change into the directory
\path{build} and execute

\begin{verbatim}
  ./run_premake.bat
\end{verbatim}

Then change into the directory \path{build/windows/vs20xx}, open the
project file with the corresponding version of Visual C++ and build
the project.

The compiled binaries will end up in the directory \path{bin}.

\input{include/gpl_v3.tex}

\end{document}


%%% Local Variables:
%%% mode: latex
%%% mode: outline-minor
%%% TeX-command-default: "Rubber"
%%% TeX-PDF-mode: t
%%% ispell-local-dictionary: "british"
%%% End:
